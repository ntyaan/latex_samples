\documentclass[dvipdfmx]{standalone}
\usepackage{tikz}
\renewcommand{\kanjifamilydefault}{\gtdefault}
\renewcommand{\familydefault}{\sfdefault}
\tikzset{
  treenode/.style = {shape=rectangle, rounded corners,
    draw, align=center,
    top color=white, bottom color=blue!20},
  root/.style     = {treenode, font=\Large, bottom color=red!30},
  env/.style      = {treenode, font=\ttfamily\normalsize},
}
\begin{document}
\tikzstyle{block} = [rectangle, draw, scale=0.15mm,
  text width=20em, text centered,node distance=2.0cm,draw=white ,rounded corners, minimum height=2em]
\begin{tikzpicture}
  [
    grow                    = right,
    sibling distance        = 6em,
    level distance          = 9em,
    edge from parent/.style = {draw, -latex},
    every node/.style       = {font=\footnotesize},
    sloped
  ]
  \node [block] (p1) at (9.3,-2) {\Large \\\textcolor{red}{赤}:クラスタリング手法\\\textcolor{blue}{青}:応用       };
  \node [root] {機械学習}
  child { node [env] (a){教師あり学習\\\footnotesize{訓練データあり}}
    child { node [env] (b){ 回帰 }
    }
    child { node [env] (c){ 分類 }
      child { node [env] (d){
          \footnotesize{・\textcolor{blue}{識別器   }}
      }}
  }}
  child { node [env] (e){教師なし学習\\\footnotesize{訓練データなし}}
    child { node [env] (f){\textcolor{black}{クラスタリング}}
      child { node [env] (g){\footnotesize{・\textcolor{red}{階層的   }}\\\footnotesize{・\textcolor{red}{非階層的  }}\\\footnotesize{・\textcolor{blue}{推薦システム}}\vspace{-1mm}
      }}
  }};
\end{tikzpicture}
\end{document}
