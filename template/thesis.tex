\documentclass[11pt,oneside,openany,uplatex,dvipdfmx]{jsbook}
\setcounter{tocdepth}{3}
\usepackage[bookmarksdepth=subsection,dvipdfmx,setpagesize=false,bookmarksnumbered]{hyperref}
\usepackage{pxjahyper}
\usepackage[chapter]{algorithm}
\usepackage{amsmath,amsthm, amssymb}
\usepackage[dvipdfmx]{color}
\usepackage{graphicx} 
\usepackage{ascmac}
\usepackage{lscape} 
\usepackage{setspace}
%\usepackage{slashbox}
\usepackage{multirow}
\usepackage{booktabs}
\usepackage{subcaption}
\usepackage{url}
%\usepackage{tikz,tikz-uml}
\usepackage{colortbl}
\usepackage{pxrubrica}
\usepackage{dcolumn} 
\usepackage{enumitem}
\usepackage{diagbox}
\usepackage[maxfloats=256]{morefloats}
\makeatletter
\def\minimize{\mathop{\operator@font~minimize}}
\def\maximize{\mathop{\operator@font~maximize}} 
\def\T#1{#1^{\top}}
\makeatother

% 余白の設定.
\usepackage[top=25truemm,bottom=15truemm,left=30truemm,right=15truemm]{geometry}
\newcommand{\NA}{---}
\newcommand{\Space}{\small{\backslashbox{  }{   }}}
\newcommand{\argmax}{\operatornamewithlimits{\mathrm{arg\,max}}}
\newcommand{\argmin}{\operatornamewithlimits{\mathrm{arg\,min}}}
\newcommand{\QED}{\hfill$\blacksquare$\par}
\theoremstyle{definition} \newtheorem{theorem}{定理}
\newtheorem*{theorem*}{定理} \newtheorem{Def}[theorem]{定義}
\newtheorem*{Def*}{定義} \newtheorem{Alg}[theorem]{アルゴリズム}
\newtheorem*{Alg*}{アルゴリズム} \newtheorem{Exp}{実験の流れ}
\newtheorem*{Exp*}{実験の流れ} 

\begin{document}

\frontmatter
\tableofcontents
\listoffigures
\listoftables

% 本文
\mainmatter
\chapter{序論}
\def\thepage{\arabic{page}}
\setcounter{page}{1}\label{chap:Intro}
\section{背景}
\label{sec:Intro_Background}
本論文では・・・.
\section{目的}
\label{sec:Intro_Purpose}
目的は・・・.
\section{構成}
\label{sec:Intro_Contents}
構成は・・・.
\chapter{結論}
\label{chap:Conculusion}
結論は・・・.
\begin{thebibliography}{99}
  \bibitem{k-means}
  MacQueen, J. B.: 
  ``Some Methods for Classification and
  Analysis of Multivariate Observations,''
  Proc. 5th Berkeley Symposium on Mathematical Statistics and Probability,
  pp.~281--297, (1967).
\end{thebibliography}
  \makeatletter 
  \renewcommand{\bibname}{代表的な研究業績}
  \renewcommand{\@biblabel}[1]{(#1)}
  \makeatother
  \begin{thebibliography}{99}
  \bibitem{1}
    Nanashi G and Yamada T.: ``Title ha Nai.''
    ○○, Vol.~1, No.~1, pp.~100--102, (1999).
  \end{thebibliography}
  \backmatter
  \chapter*{感想}
  \addcontentsline{toc}{chapter}{感想}\label{chap:feel}
  感想は・・・.
   \chapter*{謝辞}
   \addcontentsline{toc}{chapter}{謝辞}\label{chap:ack}
   感謝します.
   \chapter*{付録}
   \addcontentsline{toc}{chapter}{付録}\label{chap:program}
 \end{document}